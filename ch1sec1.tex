\documentclass{article}
\usepackage{amsmath, amssymb, amsthm}
\newtheorem*{problem}{Problem}
\newtheorem*{solution}{Solution}
\title{Chapter 1 Section 1}
\author{Andrew Taylor}
\date{March 14 2022}
\begin{document}
\maketitle

\begin{problem}
The yield of one bundle of inferior rice, two bundles of medium-grade rice, and three bundles of superior rice is 39 dou of grain. The yield of one bundle of inferior rice, three bundles of medium-grade rice, and two bundles of superior rice is 34 dou. The yield of three bundles of inferior rice, two bundles of medium-grade rice, and one bundle of superior rice is 26 dou. What is the yield of one bundle of each grade of rice?
\end{problem}

\begin{solution}
We can turn this information into an equation involving matrices.

\begin{equation*}
\begin{pmatrix}
1 & 2 & 3 \\
1 & 3 & 2 \\
3 & 2 & 1
\end{pmatrix}
\begin{pmatrix}
x \\ 
y \\
z
\end{pmatrix}
=
\begin{pmatrix}
39 \\
34 \\
26
\end{pmatrix}
\end{equation*} \\

Let's get the determinant of the coefficients matrix.

\begin{align*}
\begin{vmatrix}
1 & 2 & 3 \\
1 & 3 & 2 \\
3 & 2 & 1
\end{vmatrix}
&= 
1 \begin{vmatrix}3 & 2 \\ 2 & 1\end{vmatrix} 
- 2 \begin{vmatrix} 1 & 2 \\ 3 & 1\end{vmatrix}
+ 3 \begin{vmatrix}1 & 3 \\ 3 & 2\end{vmatrix} \\
&= 1 * (3 - 4) - 2 * (1 - 6) + 3 * (2 - 9) \\
&= -1 + 10 - 21 \\
&= -12
\end{align*}
\\
The determinant is nonzero, so the equation has a solution.
\\
\\
Now we can do some row operations to get the identity matrix on the left.

\begin{align*}
\begin{pmatrix}
1 & 2 & 3 \\
0 & 1 & -1 \\
3 & 2 & 1
\end{pmatrix}
\begin{pmatrix}
x \\ 
y \\
z
\end{pmatrix}
&=
\begin{pmatrix}
39 \\
-5 \\
26
\end{pmatrix}
\\
\begin{pmatrix}
1 & 2 & 3 \\
0 & 1 & -1 \\
0 & -4 & -8
\end{pmatrix}
\begin{pmatrix}
x \\ 
y \\
z
\end{pmatrix}
&=
\begin{pmatrix}
39 \\
-5 \\
-91
\end{pmatrix}
\\
\begin{pmatrix}
1 & 2 & 3 \\
0 & 1 & -1 \\
0 & 0 & -12
\end{pmatrix}
\begin{pmatrix}
x \\ 
y \\
z
\end{pmatrix}
&=
\begin{pmatrix}
39 \\
-5 \\
-111
\end{pmatrix}
\\
\begin{pmatrix}
1 & 2 & 3 \\
0 & 1 & -1 \\
0 & 0 & 1
\end{pmatrix}
\begin{pmatrix}
x \\ 
y \\
z
\end{pmatrix}
&=
\begin{pmatrix}
39 \\
-5 \\
9.25
\end{pmatrix}
\\
\begin{pmatrix}
1 & 2 & 3 \\
0 & 1 & 0 \\
0 & 0 & 1
\end{pmatrix}
\begin{pmatrix}
x \\ 
y \\
z
\end{pmatrix}
&=
\begin{pmatrix}
39 \\
4.25 \\
9.25
\end{pmatrix}
\\
\begin{pmatrix}
1 & 0 & 3 \\
0 & 1 & 0 \\
0 & 0 & 1
\end{pmatrix}
\begin{pmatrix}
x \\ 
y \\
z
\end{pmatrix}
&=
\begin{pmatrix}
30.5 \\
4.25 \\
9.25
\end{pmatrix}
\\
\begin{pmatrix}
1 & 0 & 0 \\
0 & 1 & 0 \\
0 & 0 & 1
\end{pmatrix}
\begin{pmatrix}
x \\ 
y \\
z
\end{pmatrix}
&=
\begin{pmatrix}
2.75 \\
4.25 \\
9.25
\end{pmatrix}
\end{align*} 
\\
The solution is $(x, y, z) = (2.75, 4.25, 9.25)$. \\ \\
Thus the yield for one bundle of inferior rice is 2.75 dou. The yield for one bundle of medium-grade rice is 4.25 dou. The yield for one bundle of superior rice is 9.25 dou.

\end{solution}

\begin{problem}
The sums of any two of three real numbers are 24, 28, and 30. Find these three numbers.
\end{problem}

\begin{solution}
This gives us the equations

\begin{align*}
x + y &= 24 \\
x + z &= 28 \\
y + z &= 30
\end{align*}

We can write this system using matrices.

\begin{align*}
\begin{pmatrix}
1 & 1 & 0 \\
1 & 0 & 1 \\
0 & 1 & 1
\end{pmatrix}
\begin{pmatrix}
x \\ y \\ z 
\end{pmatrix}
&= 
\begin{pmatrix}
24 \\ 28 \\ 30
\end{pmatrix} \\
\begin{pmatrix}
1 & 1 & 0 \\
0 & -1 & 1 \\
0 & 1 & 1
\end{pmatrix}
\begin{pmatrix}
x \\ y \\ z 
\end{pmatrix}
&= 
\begin{pmatrix}
24 \\ 4 \\ 30
\end{pmatrix} \\
\begin{pmatrix}
1 & 1 & 0 \\
0 & -1 & 1 \\
0 & 0 & 2
\end{pmatrix}
\begin{pmatrix}
x \\ y \\ z 
\end{pmatrix}
&= 
\begin{pmatrix}
24 \\ 4 \\ 34
\end{pmatrix} \\
\begin{pmatrix}
1 & 1 & 0 \\
0 & -1 & 1 \\
0 & 0 & 1
\end{pmatrix}
\begin{pmatrix}
x \\ y \\ z 
\end{pmatrix}
&= 
\begin{pmatrix}
24 \\ 4 \\ 17
\end{pmatrix} \\
\begin{pmatrix}
1 & 1 & 0 \\
0 & -1 & 0 \\
0 & 0 & 1
\end{pmatrix}
\begin{pmatrix}
x \\ y \\ z 
\end{pmatrix}
&= 
\begin{pmatrix}
24 \\ -13 \\ 17
\end{pmatrix} \\
\begin{pmatrix}
1 & 0 & 0 \\
0 & -1 & 0 \\
0 & 0 & 1
\end{pmatrix}
\begin{pmatrix}
x \\ y \\ z 
\end{pmatrix}
&= 
\begin{pmatrix}
11 \\ -13 \\ 17
\end{pmatrix} \\
\begin{pmatrix}
1 & 0 & 0 \\
0 & 1 & 0 \\
0 & 0 & 1
\end{pmatrix}
\begin{pmatrix}
x \\ y \\ z 
\end{pmatrix}
&= 
\begin{pmatrix}
11 \\ 13 \\ 17
\end{pmatrix}
\end{align*}

Thus we have $x = 11$, $y = 13$, $z = 17$.

\end{solution}

\begin{problem}
Emile and Gertrude are brother and sister. Emile has twice as many sisters as brothers, and Gertrude has just as many brothers as sisters. How many children are there in this family?
\end{problem}

\begin{solution}
This gives us the equations

\begin{align*}
S &= 2*(B-1) \\
S-1 &= B
\end{align*}

We can rearrange the terms in these equations to get S and B on one side.

\begin{align*}
S &= 2B - 2 \\
S-1 &= B \\
S - 2B &= - 2 \\
S - B &= 1
\end{align*}

Substitution gives us

\begin{align*}
1 - B &= -2 \\
B &= 3 \\
S &= 4
\end{align*}

Thus there are three sisters, four brothers, and seven children in the family.

\end{solution}

\end{document}