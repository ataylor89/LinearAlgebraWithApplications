\documentclass{article}
\usepackage{amsmath, amssymb, amsthm}
\newtheorem{problem}{Problem}
\newtheorem*{solution}{Solution}
\title{Chapter 1 Section 2}
\author{Andrew Taylor}
\date{March 16 2022}
\begin{document}
\maketitle

\begin{problem}
Find all solutions for the equations

\begin{align*}
\begin{vmatrix}
x + y - 2z = 5 \\
2x + 3y + 4z = 2
\end{vmatrix}
\end{align*}
\end{problem}

\begin{solution}

We can use Gauss-Jordan elimination.

\begin{align*}
&\begin{vmatrix}
1 & 1 & -2 & 5 \\ 
2 & 3 & 4 & 2 
\end{vmatrix} \\
&\begin{vmatrix}
1 & 1 & -2 & 5 \\ 
0 & 1 & 8 & -8 
\end{vmatrix} \\
&\begin{vmatrix}
1 & 0 & -10 & 13 \\ 
0 & 1 & 8 & -8 
\end{vmatrix}
\end{align*}

This gives us the equations $x - 10z = 13$ and $y + 8z = -8$. \\

Thus there are infinitely many solutions. The solutions are 

\begin{equation*}
\begin{pmatrix}x \\ y \\ z \end{pmatrix} = \begin{pmatrix}13 + 10t \\ -8 - 8t \\ t \end{pmatrix}
\end{equation*}

for an arbitrary real number $t$.

\end{solution}

\begin{problem}
Find all solutions for the equations

\begin{align*}
\begin{vmatrix}
3x + 4y - z = 8 \\
6x + 8y - 2z = 3
\end{vmatrix}
\end{align*}

\end{problem}

\begin{solution}

We can use Gauss-Jordan elimination once again (and after this problem we may use Gauss-Jordan elimination without saying it explicitly).

\begin{align*}
&\begin{vmatrix}
3 & 4 & -1 & 8 \\ 
6 & 8 & -2 & 3 
\end{vmatrix} \\
&\begin{vmatrix}
3 & 4 & -1 & 8 \\ 
0 & 0 & 0 & -13 
\end{vmatrix} \\
\end{align*}

This gives us the equations $3x + 4y - z = 8$ and $0 = -13$. The second equation is a contradiction. \\

Why is the second equation a contradiction? It's because each equation represents a line, and the two lines do not intersect. \\

When we do Gauss-Jordan elimination, we assume there is a solution. When this assumption is wrong, we can get a contradiction like $0 = 13$. \\

Our method shows us that the system is inconsistent, so there are no solutions.

\end{solution}

\end{document}