\documentclass{article}
\usepackage{amsmath, amssymb, amsthm, graphicx}
\usepackage[export]{adjustbox}

\title{Chapter 2 Section 3}
\author{Andrew Taylor}
\date{April 11 2022}
\newtheorem{theorem}{Theorem}
\newtheorem{problem}{Problem}
\newtheorem*{solution}{Solution}

\begin{document}
\maketitle

\begin{problem}
Calculate the matrix product

\begin{align*}
\begin{bmatrix}
6 & 7 \\
8 & 9
\end{bmatrix}
\begin{bmatrix}
1 & 2 \\
3 & 5
\end{bmatrix}
\end{align*}
\end{problem}

\begin{solution}
\begin{align*}
\begin{bmatrix}
6 & 7 \\
8 & 9
\end{bmatrix}
\begin{bmatrix}
1 & 2 \\
3 & 5
\end{bmatrix}
&= 
\begin{bmatrix}
6*1 + 7*3 & 6*2 + 7*5 \\
8*1 + 9*3 & 8*2 + 9*5
\end{bmatrix} \\
&= 
\begin{bmatrix}
27 & 47 \\
35 & 61
\end{bmatrix}
\end{align*}
\end{solution}

\begin{problem}
Compute the products BA and AB for 

\begin{align*}
A = \begin{bmatrix}0 & 1 \\ 1 & 0 \end{bmatrix}
\end{align*}

\begin{align*}
B = \begin{bmatrix}-1 & 0 \\ 0 & 1 \end{bmatrix} 
\end{align*}

Interpret your answers geometrically, as composites of linear transformation.
\end{problem}

\begin{solution}

\begin{align*}
BA &= \begin{bmatrix}-1 & 0 \\ 0 & 1 \end{bmatrix} 
\begin{bmatrix}0 & 1 \\ 1 & 0 \end{bmatrix} \\
&= \begin{bmatrix}0 & -1 \\ 1 & 0 \end{bmatrix} 
\end{align*}

The product BA is a rotation matrix that rotates a vector ninety degrees counterclockwise.

\begin{align*}
AB &= \begin{bmatrix}0 & 1 \\ 1 & 0 \end{bmatrix} 
\begin{bmatrix}-1 & 0 \\ 0 & 1 \end{bmatrix} \\
&= \begin{bmatrix}0 & 1 \\ -1 & 0 \end{bmatrix} 
\end{align*}

The product AB is a rotation matrix that rotates a vector ninety degrees clockwise.

\end{solution}

\end{document}