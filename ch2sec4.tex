\documentclass{article}
\usepackage{amsmath, amssymb, amsthm, graphicx}
\usepackage[export]{adjustbox}

\title{Chapter 2 Section 4}
\author{Andrew Taylor}
\date{April 16 2022}
\newtheorem{theorem}{Theorem}
\newtheorem{problem}{Problem}
\newtheorem*{solution}{Solution}

\begin{document}
\maketitle

\begin{theorem}
An $n \times n$ matrix A is invertible if and only if

\begin{align*}
rref(A) = I_{n}
\end{align*}

or, equivalently, if

\begin{align*}
rank(A) = n
\end{align*}

\end{theorem}

\begin{theorem}
To find the inverse of an $n \times n$ matrix A, form the $n \times (2n)$ matrix $\begin{bmatrix} A \ \vert \ I_{n} \end{bmatrix}$ and compute $rref [A \ \vert \ I_{n}]$. 

\begin{itemize}
\item If $rref [A \ \vert \ I_{n}]$ is of the form $\begin{bmatrix} I_{n} \ \vert \ B \end{bmatrix}$ then A is invertible and $A^{-1} = B$.
\item If $rref [A \ \vert \ I_{n}]$ is of another form (i.e., its left half fails to be $I_{n}$) then A is not invertible. 
\end{itemize}
\end{theorem}

\begin{theorem}
For an invertible $n \times n$ matrix A, 

\begin{align*}
A^{-1} A = I_{n} \quad \textrm{and} \quad AA^{-1} = I_{n}
\end{align*}
\end{theorem}

\begin{theorem}
If A and B are invertible $n \times n$ matrices, then BA is invertible as well, and 

\begin{align*}
(BA)^{-1} = A^{-1} B^{-1}
\end{align*}
\end{theorem}

\begin{theorem}
Let A and B be two $n \times n$ matrices such that $BA = I_{n}$. Then

\begin{itemize}
\item A and B are both invertible
\item $A^{-1} = B$ and $B^{-1} = A$
\item $AB = I_{n}$
\end{itemize}
\end{theorem}

\begin{problem}
Is the matrix

\begin{align*}
A = \begin{bmatrix}
1 & 1 & 1 \\
2 & 3 & 2 \\
3 & 8 & 2
\end{bmatrix}
\end{align*}

invertible? If so, find the inverse of $A$.
\end{problem}

\begin{solution}
\begin{align*}
& \begin{bmatrix}
1 & 1 & 1 \\
2 & 3 & 2 \\
3 & 8 & 2
\end{bmatrix} \\
& \begin{bmatrix}
1 & 1 & 1 \\
0 & 1 & 0 \\
0 & 5 & -1
\end{bmatrix} \\
& \begin{bmatrix}
1 & 1 & 1 \\
0 & 1 & 0 \\
0 & 0 & -1
\end{bmatrix} \\
& \begin{bmatrix}
1 & 1 & 1 \\
0 & 1 & 0 \\
0 & 0 & 1
\end{bmatrix} \\
& \begin{bmatrix}
1 & 0 & 0 \\
0 & 1 & 0 \\
0 & 0 & 1
\end{bmatrix}
\end{align*}

We see that 

\begin{align*}
rref(A) = \begin{bmatrix}
1 & 0 & 0 \\
0 & 1 & 0 \\
0 & 0 & 1
\end{bmatrix}
\end{align*}

Thus A is invertible. \\

Note that $rref(A)$ is an acronym that refers to the reduced row echelon form of matrix A. The computation $rref(A)$ tells us whether A is invertible. \\

To invert the matrix, let's calculate $rref \begin{bmatrix} A \ \vert \ I_{n} \end{bmatrix}$. \\

\begin{align*}
& \begin{bmatrix}
1 & 1 & 1 & \vert & 1 & 0 & 0 \\
2 & 3 & 2 & \vert & 0 & 1 & 0 \\
3 & 8 & 2 & \vert & 0 & 0 & 1
\end{bmatrix} \\
& \begin{bmatrix}
1 & 1 & 1 & \vert & 1 & 0 & 0 \\
0 & 1 & 0 & \vert & -2 & 1 & 0 \\
0 & 5 & -1 & \vert & -3 & 0 & 1
\end{bmatrix} \\
& \begin{bmatrix}
1 & 1 & 1 & \vert & 1 & 0 & 0 \\
0 & 1 & 0 & \vert & -2 & 1 & 0 \\
0 & 0 & -1 & \vert & 7 & -5 & 1
\end{bmatrix} \\
& \begin{bmatrix}
1 & 1 & 1 & \vert & 1 & 0 & 0 \\
0 & 1 & 0 & \vert & -2 & 1 & 0 \\
0 & 0 & 1 & \vert & -7 & 5 & -1
\end{bmatrix} \\
& \begin{bmatrix}
1 & 1 & 0 & \vert & 8 & -5 & 1 \\
0 & 1 & 0 & \vert & -2 & 1 & 0 \\
0 & 0 & 1 & \vert & -7 & 5 & -1
\end{bmatrix} \\
& \begin{bmatrix}
1 & 0 & 0 & \vert & 10 & -6 & 1 \\
0 & 1 & 0 & \vert & -2 & 1 & 0 \\
0 & 0 & 1 & \vert & -7 & 5 & -1
\end{bmatrix} \\
\end{align*}

Thus 

\begin{align*}
rref \begin{bmatrix} A \ \vert \ I_{n} \end{bmatrix} = \begin{bmatrix}
1 & 0 & 0 & \vert & 10 & -6 & 1 \\
0 & 1 & 0 & \vert & -2 & 1 & 0 \\
0 & 0 & 1 & \vert & -7 & 5 & -1
\end{bmatrix}
\end{align*}

and 

\begin{align*}
A^{-1} = \begin{bmatrix}
10 & -6 & 1 \\
-2 & 1 & 0 \\
-7 & 5 & -1
\end{bmatrix}
\end{align*}

\end{solution}

\begin{problem}
Suppose A, B, and C are three $n \times n$ matrices such that $ABC = I_{n}$. Show that B is invertible, and express $B^{-1}$ in terms of A and C.
\end{problem}

\begin{solution}
By the associative property of matrices

\begin{align*}
ABC &= I_{n} \\
(AB)C &= I_{n} \\
A(BC) &= I_{n} 
\end{align*}

Thus matrices A and C are invertible. 

\begin{align*}
ABC &= I_{n} \\
A^{-1}ABC &= A^{-1} I_{n} \\
BC &= A^{-1} \\
%BCC^{-1} &= A^{-1} I_{n} C^{-1} \\
%B &= A^{-1} C^{-1} \\
BCA &= A^{-1} A \\
B(CA) &= I_{n}
\end{align*}

Thus matrix B is invertible and $B^{-1} = CA$.

\end{solution}

\begin{problem}
For an arbitrary $2 \times 2$ matrix $A = \begin{bmatrix} a & b \\ c & d \end{bmatrix}$ compute the product $\begin{bmatrix}d & -b \\ -c & a \end{bmatrix} \begin{bmatrix}a & b \\ c & d \end{bmatrix}$. When is A invertible? If so, what is $A^{-1}$?
\end{problem}

\begin{solution}
\begin{align*}
& \begin{bmatrix}d & -b \\ -c & a \end{bmatrix}
\begin{bmatrix} a & b \\ c & d \end{bmatrix} \\
&= \begin{bmatrix} ad - bc & 0 \\ 0 & ad - bc \end{bmatrix}
\end{align*}

When $ad - bc$ is nonzero, we can form the product

\begin{align*}
& \displaystyle \frac{1}{ad-bc} \begin{bmatrix}d & -b \\ -c & a \end{bmatrix}
\begin{bmatrix} a & b \\ c & d \end{bmatrix} \\
&= \displaystyle \frac{1}{ad-bc} \begin{bmatrix} ad - bc & 0 \\ 0 & ad - bc \end{bmatrix} \\
&= \begin{bmatrix} 1 & 0 \\ 0 & 1 \end{bmatrix}
\end{align*}

Thus A is invertible when the determinant $ad - bc \neq 0$, and 

\begin{align*}
A^{-1} = \displaystyle \frac{1}{ad-bc} \begin{bmatrix}d & -b \\ -c & a \end{bmatrix}
\end{align*}

\end{solution}

\begin{problem}
Is the matrix $A = \begin{bmatrix}1 & 3 \\ 2 & 1\end{bmatrix}$ invertible? If so, find the inverse. Interpret det A geometrically.
\end{problem}


\end{document}