\documentclass{article}
\usepackage{amsmath, amssymb, amsthm, graphicx}
\usepackage[export]{adjustbox}

\title{Chapter 3 Section 1}
\author{Andrew Taylor}
\date{April 23 2022}
\newtheorem{theorem}{Theorem}
\newtheorem{problem}{Problem}
\newtheorem*{solution}{Solution}
\newcommand{\rref}[1]{\mathrm{rref \, #1}}
\DeclareMathOperator{\Span}{span}
\DeclareMathOperator{\Rank}{rank}
\DeclareMathOperator{\im}{im}


\begin{document}
\maketitle

\begin{problem}
Find the kernel of the linear transformation 

\begin{align*}
T(\vec{x}) = \begin{bmatrix}1 & 1 & 1 \\ 1 & 2 & 3 \end{bmatrix} \vec{x}
\end{align*}

from $\mathbb{R}^3$ to $\mathbb{R}^2$.

\end{problem}

\begin{solution}
Let's solve the linear system $T(\vec{x}) = 0$ to get the kernel of T.

\begin{align*}
& \left( \begin{array}{@{}ccc|c@{}}
1 & 1 & 1 & 0 \\
1 & 2 & 3 & 0
\end{array} \right) \\
& \left( \begin{array}{@{}ccc|c@{}}
1 & 1 & 1 & 0 \\
0 & 1 & 2 & 0
\end{array} \right) \\
& \left( \begin{array}{@{}ccc|c@{}}
1 & 0 & -1 & 0 \\
0 & 1 & 2 & 0
\end{array} \right) \\
\end{align*}

This tells us that $x_{1} = x_{3}$ and $x_{2} = -2x_{3}$. \\

Let $t = x_{3}$ be an arbitrary real number. Then the solutions to the linear system are

\begin{align*}
\ker(T) = \begin{bmatrix} t \\ -2t \\ t \end{bmatrix} = t \begin{bmatrix} 1 \\ -2 \\ 1 \end{bmatrix}
\end{align*}

The kernel of T is the line spanned by the vector $\vec{v} = \begin{bmatrix}1 \\ -2 \\ 1\end{bmatrix}$.

\end{solution}

\begin{problem}
Find the kernel of the linear transformation 

\begin{align*}
T(\vec{x}) = A\vec{x} =
\begin{bmatrix}
1 & 2 & 2 & -5 & 6 \\ 
-1 & -2 & -1 & 1 & -1 \\
4 & 8 & 5 & -8 & 9 \\ 
3 & 6 & 1 & 5 & -7 
\end{bmatrix} \begin{bmatrix} x_{1} \\ x_{2} \\ x_{3} \\ x_{4} \end{bmatrix}
\end{align*}

from $\mathbb{R}^5$ to $\mathbb{R}^4$.

\end{problem}

\begin{solution}
Let's solve the linear system $T(\vec{x}) = Ax = 0$. \\

We can solve this linear system by creating the augmented matrix $\begin{bmatrix}A & \vert & \vec{0} \end{bmatrix}$ and calculating $\rref{\begin{bmatrix}A & \vert & \vec{0} \end{bmatrix}}$.

\begin{align*}
&\left( \begin{array}{@{}ccccc|c@{}}
1 & 2 & 2 & -5 & 6 & 0 \\ 
-1 & -2 & -1 & 1 & -1 & 0 \\
4 & 8 & 5 & -8 & 9 & 0 \\ 
3 & 6 & 1 & 5 & -7 & 0 
\end{array} \right) \\
&\left( \begin{array}{@{}ccccc|c@{}}
0 & 0 & 1 & -4 & 5 & 0 \\ 
-1 & -2 & -1 & 1 & -1 & 0 \\
4 & 8 & 5 & -8 & 9 & 0 \\ 
3 & 6 & 1 & 5 & -7 & 0 
\end{array} \right) \textrm{Adding row2 to row1} \\
&\left( \begin{array}{@{}ccccc|c@{}}
0 & 0 & 1 & -4 & 5 & 0 \\ 
-1 & -2 & -1 & 1 & -1 & 0 \\
0 & 0 & 1 & -4 & 5 & 0 \\ 
3 & 6 & 1 & 5 & -7 & 0 
\end{array} \right) \textrm{Adding row2 to row1} \\
&\left( \begin{array}{@{}ccccc|c@{}}
0 & 0 & 0 & 0 & 0 & 0 \\ 
-1 & -2 & -1 & 1 & -1 & 0 \\
0 & 0 & 1 & -4 & 5 & 0 \\ 
3 & 6 & 1 & 5 & -7 & 0 
\end{array} \right) \textrm{Subtracting row3 from row1} \\
&\left( \begin{array}{@{}ccccc|c@{}}
0 & 0 & 0 & 0 & 0 & 0 \\ 
-1 & -2 & -1 & 1 & -1 & 0 \\
0 & 0 & 1 & -4 & 5 & 0 \\ 
0 & 0 & -2 & 8 & -10 & 0 
\end{array} \right) \textrm{Adding row2 to row4} \\
&\left( \begin{array}{@{}ccccc|c@{}}
0 & 0 & 0 & 0 & 0 & 0 \\ 
-1 & -2 & -1 & 1 & -1 & 0 \\
0 & 0 & 1 & -4 & 5 & 0 \\ 
0 & 0 & 0 & 0 & 0 & 0 
\end{array} \right) \textrm{Adding row2 to row4} \\
&\left( \begin{array}{@{}ccccc|c@{}}
0 & 0 & 0 & 0 & 0 & 0 \\ 
-1 & -2 & 0 & -3 & 4 & 0 \\
0 & 0 & 1 & -4 & 5 & 0 \\ 
0 & 0 & 0 & 0 & 0 & 0 
\end{array} \right) \textrm{Adding row3 to row2} \\
&\left( \begin{array}{@{}ccccc|c@{}}
1 & 2 & 0 & 3 & -4 & 0 \\
0 & 0 & 1 & -4 & 5 & 0 \\ 
0 & 0 & 0 & 0 & 0 & 0 \\
0 & 0 & 0 & 0 & 0 & 0 \\ 
\end{array} \right) \textrm{Swapping rows and multiplying row1 by -1} \\
\end{align*}

\begin{align*}
x_{1} + 2x_{2} + 3x_{4} - 4x_{5} = 0 \\
x_{3} - 4x_{4} + 5x_{5} = 0 \\
\end{align*}

\begin{align*}
x_{1} &= -2x_{2} - 3x_{4} + 4x_{5} \\
x_{3} &= 4x_{4} - 5x_{5}
\end{align*}

Let $r = x_{2}$, let $s = x_{4}$, let $t = x_{5}$. \\

Then the solutions to the linear system (the kernel) are of the form:

\begin{align*}
\ker(T) &= \begin{bmatrix}
-2r - 3s + 4t \\
r \\
4s - 5t \\
s \\
t
\end{bmatrix} \\
&=
\begin{bmatrix} -2r \\ r \\ 0 \\ 0 \\ 0 \end{bmatrix} +
\begin{bmatrix} -3s \\ 0 \\ 4s \\ s \\ 0 \end{bmatrix} +
\begin{bmatrix} 4t \\ 0 \\ -5t \\ 0 \\ t \end{bmatrix} \\
&=
r \begin{bmatrix} -2 \\ 1 \\ 0 \\ 0 \\ 0 \end{bmatrix} +
s \begin{bmatrix} -3 \\ 0 \\ 4 \\ 1 \\ 0 \end{bmatrix} +
t \begin{bmatrix} 4 \\ 0 \\ -5 \\ 0 \\ 1 \end{bmatrix} \\
&=
\Span \left( \begin{bmatrix} -2 \\ 1 \\ 0 \\ 0 \\ 0 \end{bmatrix},
\begin{bmatrix} -3 \\ 0 \\ 4 \\ 1 \\ 0 \end{bmatrix},
\begin{bmatrix} 4 \\ 0 \\ -5 \\ 0 \\ 1 \end{bmatrix} \right)
\end{align*}

\end{solution}

\begin{problem}
For an invertible $n \times n$ matrix find $\ker A$.
\end{problem}

\begin{solution}
Let $T(\vec{x}) = A\vec{x}$ be a linear transformation with an invertible $n \times n$ matrix A. We know that T is invertible because A is invertible. Thus there can only be one unique solution to the equation $A\vec{x} = 0$. Since $\vec{x} = \vec{0}$ is the unique solution, we know that $\ker A = \{\vec{0}\}$.
\end{solution}

\begin{problem}
For which $n \times m$ matrices is $\ker A = \{\vec{0}\}$. Give your answer in terms of the rank of $A$.
\end{problem}

\begin{solution}
Let $A$ be a $n \times m$ matrix. When $\Rank A = m$, we get the unique solution $\vec{x} = \vec{0}$. Thus $\ker A = \{ \vec{0} \}$ when $\Rank A = m$.
\end{solution}

In the following problems, find vectors that span the kernel of A and the image of A. 

\begin{problem}
\begin{align*}
A = \begin{bmatrix}1 & 2 \\ 3 & 4 \end{bmatrix}
\end{align*}
\end{problem}

\begin{solution}
Let $T(\vec{x}) = A\vec{x}$. Let's solve the equation $A\vec{x} = \vec{0}$.
\begin{align*}
&\left[\begin{array}{cc|c}1 & 2 & 0 \\ 3 & 4 & 0\end{array}\right] & \textrm{Augmented matrix } \left[A \quad \vert \quad \vec{0}\right] \\
&\left[\begin{array}{cc|c}1 & 2 & 0 \\ 0 & -2 & 0\end{array}\right] & \textrm{Subtract 3 times row1 from row2} \\
&\left[\begin{array}{cc|c}1 & 0 & 0 \\ 0 & -2 & 0\end{array}\right] & \textrm{Add row2 to row1} \\
&\left[\begin{array}{cc|c}1 & 0 & 0 \\ 0 & 1 & 0\end{array}\right] & \textrm{Divide row2 by -2}
\end{align*}

The solution set is $\vec{x} = \vec{0}$, thus $\ker A = \{0\}$. \\

The kernel of A is spanned by the zero vector. The image of A is the span of the column vectors $\begin{bmatrix}1 \\ 3 \end{bmatrix}$ and $\begin{bmatrix}2 \\ 4 \end{bmatrix}$. \\

In other words, $T(\vec{x}) = x_{1} \begin{bmatrix}1 \\ 3 \end{bmatrix} + x_{2} \begin{bmatrix}2 \\ 4 \end{bmatrix}$.
\end{solution}

\begin{problem}
\begin{align*}
A = \begin{bmatrix}1 & 2 & 3 \end{bmatrix}
\end{align*}
\end{problem}

\begin{solution}
Let's solve the equation
\begin{align*}
x_{1} + 2x_{2} + 3x_{3} = 0
\end{align*}

From inspection, we can find two nonparallel vectors in the solution set, the vectors 
\begin{align*}
\vec{v_{1}} = \begin{bmatrix}1 \\ 1 \\ -1\end{bmatrix} \vec{v_{2}} = \begin{bmatrix}-1 \\ 2 \\ -1\end{bmatrix}
\end{align*}

There is no constant $c$ such that $\vec{v_{2}} = c\vec{v_{1}}$, thus the two vectors are linearly independent and form a basis for the kernel. \\

Thus the kernel of A is the plane through the origin spanned by the vectors $\vec{v_{1}}$ and $\vec{v_{2}}$.

\begin{align*}
\ker A = \Span \left( \begin{bmatrix}1 \\ 1 \\ -1\end{bmatrix}, \begin{bmatrix}-1 \\ 2 \\ -1\end{bmatrix} \right)
\end{align*}

The image of A is $\mathbb{R}$, since for any $y \in \mathbb{R}$, we have the solution $\vec{x} = \begin{bmatrix} y \\ 0 \\ 0 \end{bmatrix}$.

\end{solution}

\begin{problem}
\begin{align*}
A = \begin{bmatrix}0 & 0 \\ 0 & 0 \end{bmatrix}
\end{align*}
\end{problem}

\begin{solution}
The kernel of A is $\mathbb{R}^2$ and the image of A is $\vec{0} = \begin{bmatrix} 0 \\ 0 \end{bmatrix}$. We can write the kernel of A as the span of the unit vectors $e_{1}$ and $e_{2}$.
\begin{align*}
\ker A = \Span \left( \begin{bmatrix}1 \\ 0 \end{bmatrix}, \begin{bmatrix}0 \\ 1 \end{bmatrix} \right)
\end{align*}
\end{solution}

\begin{problem}
\begin{align*}
A = \begin{bmatrix}2 & 3 \\ 6 & 9 \end{bmatrix}
\end{align*}
\end{problem}

\begin{solution}

The image is the span of the column vectors. Since the second column vector is redundant, the image of A is the line spanned by the vector $\begin{bmatrix}2 & 3 \end{bmatrix}$. \\

The kernel of A is the solution set of the equation $2x_{1} + 3x_{2} = 0$.

\begin{align*}
\displaystyle x_{1} = -\frac{3}{2}x_{2}
\end{align*}

This equation describes a line that passes through the origin. \\

The kernel of A is the line spanned by the vector $\begin{bmatrix}3 \\ -2 \end{bmatrix}$.
\end{solution}

\begin{problem}
\begin{align*}
A = \begin{bmatrix}1 & 1 & 1 \\ 1 & 2 & 3 \\ 1 & 3 & 5 \end{bmatrix}
\end{align*}
\end{problem}

\begin{solution}
Let's solve the linear system $A\vec{x} = 0$ to get the kernel of A.

\begin{align*}
& \left[ \begin{array}{ccc|c}1 & 1 & 1 & 0 \\ 1 & 2 & 3 & 0 \\ 1 & 3 & 5 & 0 \end{array} \right] \\
& \left[ \begin{array}{ccc|c}1 & 1 & 1 & 0 \\ 0 & 1 & 2 & 0 \\ 1 & 3 & 5 & 0 \end{array} \right] \\
& \left[ \begin{array}{ccc|c}1 & 1 & 1 & 0 \\ 0 & 1 & 2 & 0 \\ 0 & 2 & 4 & 0 \end{array} \right] \\
& \left[ \begin{array}{ccc|c}1 & 1 & 1 & 0 \\ 0 & 1 & 2 & 0 \\ 0 & 0 & 0 & 0 \end{array} \right] \\
& \left[ \begin{array}{ccc|c}1 & 0 & -1 & 0 \\ 0 & 1 & 2 & 0 \\ 0 & 0 & 0 & 0 \end{array} \right]
\end{align*}

This gives us the equations 

\begin{align*}
& x_{1} = x_{3} \\
& x_{2} = -2x_{3}
\end{align*}

Let $t$ be an arbitrary real number. Then we can write the kernel of A as

\begin{align*}
\ker A &= \Bigg\{ \begin{bmatrix} t \\ -2t \\ t \end{bmatrix} \Bigg\} \\
&= \Bigg\{ t \begin{bmatrix} 1 \\ -2 \\ 1 \end{bmatrix} \Bigg\} \\
&= \Span \left( \begin{bmatrix} 1 \\ -2 \\ 1 \end{bmatrix} \right)
\end{align*}

Thus the kernel of A is the line spanned by the vector $\begin{bmatrix} 1 \\ -2 \\ 1 \end{bmatrix}.$ \\

In our matrix A, the third column vector is redundant. Thus 

\begin{align*}
\im A = \Span \left( \begin{bmatrix} 1 \\ 1 \\ 1 \end{bmatrix}, \begin{bmatrix} 1 \\ 2 \\ 3 \end{bmatrix} \right)
\end{align*} 

The image of A is a plane spanned by the above vectors that passes through the origin.

\end{solution}

\begin{problem}
\begin{align*}
A = \begin{bmatrix}1 & 1 & 1 \\ 1 & 2 & 3 \end{bmatrix}
\end{align*}
\end{problem}

\begin{solution}
Let's solve the linear system $A\vec{x} = 0$.

\begin{align*}
\left[ \begin{array}{ccc|c}1 & 1 & 1 & 0 \\ 1 & 2 & 3 & 0 \end{array} \right] \\
\left[ \begin{array}{ccc|c}1 & 1 & 1 & 0 \\ 0 & 1 & 2 & 0 \end{array} \right] \\
\left[ \begin{array}{ccc|c}1 & 0 & -1 & 0 \\ 0 & 1 & 2 & 0 \end{array} \right] \\
\end{align*}

This gives us the equations 

\begin{align*}
& x_{1} = x_{3} \\
& x_{2} = -2x_{3}
\end{align*}

Let $t$ be an arbitrary real number. Then we can write the kernel of A as

\begin{align*}
\ker A &= \Bigg\{ \begin{bmatrix} t \\ -2t \\ t \end{bmatrix} \Bigg\} \\
&= \Bigg\{ t \begin{bmatrix} 1 \\ -2 \\ 1 \end{bmatrix} \Bigg\} \\
&= \Span \left( \begin{bmatrix} 1 \\ -2 \\ 1 \end{bmatrix} \right)
\end{align*}

Thus the kernel of A is the line spanned by the vector $\begin{bmatrix} 1 \\ -2 \\ 1 \end{bmatrix}.$ \\

In our matrix A, the third column vector is redundant. Thus 

\begin{align*}
\im A = \Span \left( \begin{bmatrix} 1 \\ 1 \end{bmatrix}, \begin{bmatrix} 1 \\ 2  \end{bmatrix} \right)
\end{align*} 

The image of A is a plane that passes through the origin.

\end{solution}

\end{document}