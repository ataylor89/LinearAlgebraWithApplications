\documentclass{article}
\usepackage{amsmath, amssymb, amsthm, graphicx}
\usepackage[export]{adjustbox}

\title{Chapter 3 Section 2}
\author{Andrew Taylor}
\date{April 30 2022}
\newtheorem{theorem}{Theorem}
\newtheorem{problem}{Problem}
\newtheorem*{solution}{Solution}
\newcommand{\rref}[1]{\mathrm{rref \, #1}}
\DeclareMathOperator{\Span}{span}
\DeclareMathOperator{\Rank}{rank}
\DeclareMathOperator{\im}{im}

\begin{document}
\maketitle

\begin{problem}
Is $W = \Bigg \{ \begin{bmatrix} x \\ y \end{bmatrix}$ in $\mathbb{R}^2$: $x \geq 0$ and $y \geq 0 \Bigg \}$ a subspace of $\mathbb{R}^2$?
\end{problem}

\begin{solution} 
W contains the zero vector and is closed under addition. But W is not closed under scalar multiplication. Therefore W is not a subspace of $\mathbb{R}^2$.
\end{solution}

\begin{problem}
Show that the only subspaces of $\mathbb{R}^2$ are $\mathbb{R}^2$ itself, the set $\{ \vec{0} \}$, and any of the lines through the origin.
\end{problem}

\begin{solution}
Let W be a subspace of $\mathbb{R}^2$ that is neither a line through the origin nor the set $\{ \vec{0} \}$. Then we can choose two nonzero nonparallel vectors $\vec{v} = \begin{bmatrix} v_{1} \\ v_{2} \end{bmatrix}$ and $\vec{w} = \begin{bmatrix} w_{1} \\ w_{2} \end{bmatrix}$ from our subspace W.  Let $\vec{u} = \begin{bmatrix} u_{1} \\ u_{2} \end{bmatrix}$ be a vector in $\mathbb{R}^2$. We will show that we can write $\vec{u}$ as a linear combination of $\vec{v}$ and $\vec{w}$. \\

If $\vec{u}$ can be written as a linear combination of $\vec{v}$ and $\vec{w}$, then there are solutions to the equation

\begin{align*}
x_{1} \vec{v} + x_{2} \vec{w} = \vec{u}
\end{align*}

where $x_{1}$ and $x_{2}$ are real numbers. We can write this equation in matrix form

\begin{align*}
\begin{bmatrix} v_{1} & w_{1} \\ v_{2} & w_{2} \end{bmatrix} \begin{bmatrix} x_{1} \\ x_{2} \end{bmatrix} = \begin{bmatrix} u_{1} \\ u_{2} \end{bmatrix} 
\end{align*}

This equation has solutions when $A = \begin{bmatrix} v_{1} & w_{1} \\ v_{2} & w_{2} \end{bmatrix}$ is invertible. We know that A is invertible when $\det A$ is nonzero. \\

The components $v_{1}, v_{2}, w_{1}, w_{2}$ can either be zero or nonzero. There is a small number of possible cases, since both the vectors are not the zero vector, and since the two 
vectors are not parallel. 
\\ \\
\emph{Case 1: } $v_{1} = 0, v_{2} \neq 0, w_{1} \neq 0, w_{2} = 0$ \\
\emph{Case 2: } $v_{1} \neq 0, v_{2} = 0, w_{1} = 0, w_{2} \neq 0$ \\

In both of these cases, the two vectors are scalar multiples of the unit vectors, and it easy to write any vector $\vec{u}$ as a linear combination of $\vec{v}$ and $\vec{w}$. \\
\\
\emph{Case 3: } Either $\vec{v}$ or $\vec{w}$ has two nonzero components. \\

Let $\vec{v}$ be the vector with two nonzero components. \\

There exist real numbers $c_{1}$ and $c_{2}$ such that $c_{1} v_{1} = w_{1}$ and $c_{2} v_{2} = w_{2}$. We know that $c_{1} \neq c_{2}$ since the two vectors are not scalar multiples of each other. We can substitute these expressions when we calculate the determinant of A.

\begin{align*}
\det A &= v_{1} w_{2} - v_{2} w_{1} \\
&= v_{1} (c_{2} v_{2}) - v_{2} (c_{1} v_{1}) \\
&= c_{2} v_{1} v_{2} - c_{1} v_{1} v_{2} \\
&= v_{1} v_{2}  (c_{2} - c_{1}) 
\end{align*} 

Since $v_{1} \neq 0$, $v_{2} \neq 0$ and $c_{2} \neq c_{1}$, the determinant of A is nonzero. Thus the matrix A is invertible, and the equation 

\begin{align*}
x_{1} \vec{v} + x_{2} \vec{w} = \vec{u}
\end{align*}

has solutions for $x_{1}$ and $x_{2}$. \\

Since W is closed under linear combinations, the vector $\vec{u}$ is in the subspace W. This means that W contains every real number, so $W = \mathbb{R}^2$. \\

We can also express this using a linear transformation. Let $T : \mathbb{R}^2 \to \mathbb{R}^2$ be the linear transformation.

\begin{align*}
T(\vec{x}) &= \begin{bmatrix} \vec{v} & \vec{w} \end{bmatrix} \begin{bmatrix} x_{1} \\ x_{2} \end{bmatrix} \\
&= \begin{bmatrix} v_{1} & w_{1} \\ v_{2} & w_{2} \end{bmatrix} \begin{bmatrix} x_{1} \\ x_{2} \end{bmatrix}
\end{align*}

We have shown that the matrix $\begin{bmatrix} v_{1} & w_{1} \\ v_{2} & w_{2} \end{bmatrix}$ is invertible. This means that T is invertible, and that T is a bijection, and that the image of T is $\mathbb{R}^2$. \\

This is equivalent to saying any vector $\vec{u}$ in $\mathbb{R}^2$ can be written as a linear combination of $\vec{v}$ and $\vec{w}$.

\end{solution}

\end{document}