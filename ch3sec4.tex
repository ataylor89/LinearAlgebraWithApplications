\documentclass{article}
\usepackage{amsmath, amssymb, amsthm, graphicx}
\usepackage[export]{adjustbox}

\title{Chapter 3 Section 4}
\author{Andrew Taylor}
\date{May 16 2022}
\newtheorem{definition}{Definition}
\newtheorem{theorem}{Theorem}
\newtheorem{problem}{Problem}
\newtheorem*{solution}{Solution}
\newcommand{\rref}[1]{\mathrm{rref \, #1}}
\newcommand{\im}[1]{\mathrm{im \, #1}}
\newcommand{\Span}[1]{\mathrm{span(#1)}}
\newcommand{\Rank}[1]{\mathrm{rank(#1)}}
\newcommand{\Coord}[2]{\Bigg[ \, \vec{#1} \, \Bigg]_{\mathfrak{#2}}}
\newcommand{\Proj}[2]{\mathrm{proj}_{#1}(\vec{#2})}
\begin{document}
\maketitle

\begin{problem}
Let $\vec{v_{1}} = \begin{bmatrix} 1 \\ 1 \\ 1 \end{bmatrix}$ and $\vec{v_{2}} = \begin{bmatrix} 1 \\ 2 \\ 3 \end{bmatrix}$. Let $V = \Span{v_{1}, v_{2}}$. Is the vector $\vec{w} = \begin{bmatrix} 5 \\ 7 \\ 9 \end{bmatrix}$ on the plane V?
\end{problem}

\begin{solution}
If the vector $\vec{w}$ is on the plane V, then there exist some $x_{1}, x_{2} \in \mathbb{R}$ such that $\vec{w} = x_{1} \vec{v_{1}} + x_{2} \vec{v_{2}}$. This gives us the equations

\begin{align*}
x_{1} + x_{2} &= 5 \\
x_{1} + 2x_{2} &= 7 \\ 
x_{1} + 3x_{2} &= 9
\end{align*}

We can solve these equations using a matrix.

\begin{align*}
\left[ \begin{array}{cc|c}
1 & 1 & 5 \\
1 & 2 & 7 \\
1 & 3 & 9
\end{array} \right] \\
\left[ \begin{array}{cc|c}
1 & 1 & 5 \\
0 & 1 & 2 \\
0 & 2 & 4
\end{array} \right] \\
\left[ \begin{array}{cc|c}
1 & 1 & 5 \\
0 & 1 & 2 \\
0 & 0 & 0
\end{array} \right] \\
\left[ \begin{array}{cc|c}
1 & 0 & 3 \\
0 & 1 & 2 \\
0 & 0 & 0
\end{array} \right]
\end{align*}

This gives us $x_{1} = 3$ and $x_{2} = 2$. \\

Thus $\vec{w}$ is on the plane V because $\vec{w} = 3 \vec{v_{1}} + 2 \vec{v_{2}}$.

\end{solution}

\begin{problem}
Consider the basis $\mathfrak{B}$ of $\mathbb{R}^2$ consisting of the vectors $\vec{v_{1}} = \begin{bmatrix}3 \\ 1 \end{bmatrix}$ and $\vec{v_{2}} = \begin{bmatrix}-1 \\ 3 \end{bmatrix}$. 

\begin{itemize}
\item If $\vec{x} = \begin{bmatrix} 10 \\ 10 \end{bmatrix}$ find $\Coord{x}{B}$
\item If $\Coord{y}{B} = \begin{bmatrix} 2 \\ -1 \end{bmatrix}$ find $\vec{y}$
\end{itemize}
\end{problem}

\begin{solution}
We can find the coordinates of $\vec{x}$ with respect to $\mathfrak{B}$ by means of an equation.

\begin{align*}
\vec{x} &= c_{1} \vec{v_{1}} + c_{2} \vec{v_{2}} \\
\begin{bmatrix} 10 \\ 10 \end{bmatrix} &= c_{1} \begin{bmatrix}3 \\ 1 \end{bmatrix} + c_{2} \begin{bmatrix} -1 \\ 3 \end{bmatrix} \\
\begin{bmatrix} 10 \\ 10 \end{bmatrix} &= \begin{bmatrix}3 & -1 \\ 1 & 3 \end{bmatrix} \begin{bmatrix} c_{1} \\ c_{2} \end{bmatrix} \\
\end{align*}

We can solve this equation using elementary row operations.

\begin{align*}
\left( \begin{array}{cc|c}
3 & -1 & 10 \\
1 & 3 & 10
\end{array} \right) \\
\left( \begin{array}{cc|c}
0 & -10 & -20 \\
1 & 3 & 10
\end{array} \right) \\
\left( \begin{array}{cc|c}
0 & 1 & 2 \\
1 & 3 & 10
\end{array} \right) \\
\left( \begin{array}{cc|c}
0 & 1 & 2 \\
1 & 0 & 4
\end{array} \right) \\
\left( \begin{array}{cc|c}
1 & 0 & 4 \\
0 & 1 & 2 
\end{array} \right) \\
\end{align*}

By reducing the matrix, we find that $c_{1}  = 4$ and $c_{2} = 2$. Thus $\Coord{x}{B} = \begin{bmatrix} 4 \\ 2 \end{bmatrix}$. \\

We can also use an equation to solve for $\vec{y}$.

\begin{align*}
\vec{y} &= 2 \vec{v_{1}} - \vec{v_{2}} \\
&= 2 \begin{bmatrix}3 \\ 1 \end{bmatrix} - \begin{bmatrix} -1 \\ 3 \end{bmatrix} \\
&= \begin{bmatrix} 6 \\ 2 \end{bmatrix} - \begin{bmatrix} -1 \\ 3 \end{bmatrix} \\
&= \begin{bmatrix} 7 \\ -1 \end{bmatrix}
\end{align*}
\end{solution}

\begin{problem}
Let $\vec{v_{1}}$ and $\vec{v_{2}}$ be perpendicular unit vectors in $\mathbb{R}^3$. Let $\vec{v_{3}}$ be the cross product of $\vec{v_{1}}$ and $\vec{v_{2}}$, that is, $\vec{v_{3}} = \vec{v_{1}} \times \vec{v_{2}}$. We know from the properties of the cross product that $\vec{v_{3}}$ is perpendicular to $\vec{v_{1}}$ and $\vec{v_{2}}$. Thus the three vectors are linearly independent. The three vectors form a basis for $\mathbb{R}^3$.

\begin{enumerate}
\item What is $\vec{v_{1}} \times \vec{v_{3}}$?
\item Find the $\mathfrak{B}$-matrix of the linear transformation $T(x) = \vec{v_{1}} \times \vec{x}$.
\end{enumerate}
\end{problem}

\begin{solution}
$\vec{v_{1}} \times \vec{v_{3}} = -\vec{v_{2}}$. \\

The $\mathfrak{B}$-matrix is the matrix B such that \\

\begin{align*}
\Coord{T(x)}{B} = B \Coord{x}{B}
\end{align*}

%We know that $\Coord{x}{B} = \begin{bmatrix} \vec{x} \cdot \vec{v_{1}} \\ \vec{x} \cdot \vec{v_{2}} \\ \vec{x} \cdot \vec{v_{3}}\end{bmatrix}$.

We can find the coordinates of $\vec{x}$ with respect to $\mathfrak{B}$ by means of an equation.

\begin{align*}
\vec{x} &= c_{1} \vec{v_{1}} + c_{2} \vec{v_{2}} + c_{3} \vec{v_{3}} \\
&= \begin{bmatrix} \vec{v_{1}} & \vec{v_{2}} & \vec{v_{3}} \end{bmatrix} \begin{bmatrix} c_{1} \\ c_{2} \\ c_{3} \end{bmatrix}
\end{align*}
\end{solution}

Likewise, we have

\begin{align*}
T(x) &= \vec{v_{1}} \times \vec{x} \\
&= \vec{v_{1}} \times (c_{1} \vec{v_{1}} + c_{2} \vec{v_{2}} + c_{3} \vec{v_{3}}) \\
&= c_{1} (\vec{v_{1}} \times \vec{v_{1}}) + c_{2} (\vec{v_{1}} \times \vec{v_{2}}) + c_{3} (\vec{v_{1}} \times \vec{v_{3}}) \\
&= c_{2} v_{3} - c_{3} \vec{v_{2}} \\
&= \begin{bmatrix} \vec{v_{1}} & \vec{v_{2}} & \vec{v_{3}} \end{bmatrix} \begin{bmatrix} 0 \\ -c_{3} \\ c_{2} \end{bmatrix}
\end{align*}

Thus 

\begin{align*}
\Coord{x}{B} = \begin{bmatrix} c_{1} \\ c_{2} \\ c_{3} \end{bmatrix}
\end{align*} 

and 

\begin{align*}
\Coord{T(x)}{B} = \begin{bmatrix} 0 \\ -c_{3} \\ c_{2} \end{bmatrix}
\end{align*} 

Now let's find the matrix B such that 

\begin{align*}
\begin{bmatrix} 0 \\ -c_{3} \\ c_{2} \end{bmatrix} = B \begin{bmatrix} c_{1} \\ c_{2} \\ c_{3} \end{bmatrix}
\end{align*}

By inspection, we see that

\begin{align*}
B = \begin{bmatrix} 0 & 0 & 0 \\ 0 & 0 & -1 \\ 0 & 1 & 0 \end{bmatrix}
\end{align*}

Thus 

\begin{align*}
\Coord{T(x)}{B} = \begin{bmatrix} 0 & 0 & 0 \\ 0 & 0 & -1 \\ 0 & 1 & 0 \end{bmatrix} \Coord{x}{B}
\end{align*}

and the $\mathfrak{B}$-matrix of T is

\begin{align*}
B = \begin{bmatrix} 0 & 0 & 0 \\ 0 & 0 & -1 \\ 0 & 1 & 0 \end{bmatrix}
\end{align*}

\begin{problem}
Let T be the linear transformation from $\mathbb{R}^2$ to $\mathbb{R}^2$ that projects any vector onto the line L spanned by the vector $\begin{bmatrix} 3 \\ 1 \end{bmatrix}$. Earlier we found that the $\mathfrak{B}$-matrix of T with respect to the basis $\mathfrak{B} = \left( \begin{bmatrix} 3 \\ 1 \end{bmatrix}, \begin{bmatrix} -1 \\ 3 \end{bmatrix} \right)$ is

\begin{align*}
B = \begin{bmatrix} 1 & 0 \\ 0 & 0 \end{bmatrix}
\end{align*}

What is the relationship between B and the standard matrix A of T (such that T(x) = Ax)?
\end{problem}

\begin{solution}
\begin{align*}
\Proj{L}{x} &= \dfrac{1}{w_{1}^2 + w_{2}^2} \begin{bmatrix} w_{1}^2 & w_{1}w_{2} \\ w_{1}w_{2} & w_{2}^2 \end{bmatrix} \\
&= \dfrac{1}{3^2 + 1^2} \begin{bmatrix} 3^2 & 3*1 \\ 3*1 & 1^2 \end{bmatrix} \\
&= \dfrac{1}{10} \begin{bmatrix} 9 & 3 \\ 3 & 1 \end{bmatrix} \\
&= \begin{bmatrix} \dfrac{9}{10} & \dfrac{3}{10} \\ \\ \dfrac{3}{10} & \dfrac{1}{10} \end{bmatrix} \\
&= \begin{bmatrix} 0.9 & 0.3 \\ 0.3 & 0.1 \end{bmatrix}
\end{align*}

Thus the standard matrix of T is 

\begin{align*}
A = \begin{bmatrix} 0.9 & 0.3 \\ 0.3 & 0.1 \end{bmatrix}
\end{align*}

and 

\begin{align*}
T(\vec{x}) &= A \begin{bmatrix} x_{1} \\ x_{2} \end{bmatrix} \\
&= \begin{bmatrix} 0.9 & 0.3 \\ 0.3 & 0.1 \end{bmatrix} \begin{bmatrix} x_{1} \\ x_{2} \end{bmatrix}
\end{align*}

Now we're going to find the relationship between A and B. \\

Let $S = \begin{bmatrix} 3 & -1 \\ 1 & 3 \end{bmatrix}$ \\

Let's write $T(\vec{x})$ in terms of A and S.

\begin{align*}
\vec{x} &= S \Coord{x}{B} \\
T(\vec{x}) &= A \vec{x} \\
T(\vec{x}) &= A S \Coord{x}{B} 
\end{align*}

Now let's write $T(\vec{x})$ in terms of B and S.

\begin{align*}
\Coord{T(x)}{B} &= B\Coord{x}{B} \\
T(\vec{x}) &= S \Coord{T(x)}{B}  \\
T(\vec{x}) &= S B\Coord{x}{B}
\end{align*}

The above equations show that $AS = SB$ and $A = SBS^{-1}$. \\

The equation $A = SBS^{-1}$ gives us another way of finding A (since we know S, we know B, and we can calculate $S^{-1}$).

\end{solution}

\begin{definition}
Two $n \times n$ matrices A and B are similar if there exists an invertible matrix S such that

\begin{align*}
AS = SB\textrm{, or }B = S^{-1}AS
\end{align*}

\end{definition}

\begin{problem}
Is matrix $A = \begin{bmatrix}1 & 2 \\ 4 & 3 \end{bmatrix}$ similar to $B = \begin{bmatrix}5 & 0 \\ 0 & -1 \end{bmatrix}$?
\end{problem}

\begin{solution}
We're looking for a matrix $S = \begin{bmatrix} x & y \\ z & t \end{bmatrix}$ such that

\begin{align*}
\begin{bmatrix}x + 2z & y + 2t \\ 4x + 3z & 4y + 3t\end{bmatrix} = \begin{bmatrix}5x & -y \\ 5z & -t\end{bmatrix}
\end{align*}

By inspection we see that $z = 2x$ and $t = -y$. Therefore

\begin{align*}
S = \begin{bmatrix} x & y \\ 2x & -y \end{bmatrix} 
\end{align*}

Now let's look at the determinant of S.

\begin{align*}
det(S) = -3xy
\end{align*}

The matrix S is invertible when $x \neq 0$ and $y \neq 0$. Thus 

\begin{align*}
S = \begin{bmatrix} x & y \\ 2x & -y \end{bmatrix} 
\end{align*}

where $x \neq 0$ and $y \neq 0$. \\

We have found invertible matrices S such that $AS = SB$, so we know that matrix A is similar to matrix B. 

\end{solution}

\begin{theorem}
If matrix A is similar to matrix B, then its power $A^{t}$ is similar to $B^{t}$ for all positive integers t.
\end{theorem}

\begin{proof}
Let matrix A be similar to matrix B. Then there exists an invertible matrix S such that 

\begin{align*}
AS &= SB \\
B &= S^{-1}AS
\end{align*}

When we simplify the expression for $B^{t}$, most of the $S^{-1}$ and $S$ terms cancel.

\begin{align*}
B^{t} &= (S^{-1}AS)^{t} \\
&= (S^{-1}AS)(S^{-1}AS)\cdots(S^{-1}AS) \\
&= S^{-1}A^{t}S
\end{align*}

Arriving at the equation $B^{t} = S^{-1}A^{t}S$, we have proven that $B^{t}$ is similar to $A^{t}$ for all positive integers t.

\end{proof}

\end{document}