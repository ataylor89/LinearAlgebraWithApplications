\documentclass{article}
\usepackage{amsmath, amssymb, amsthm, graphicx}
\usepackage[export]{adjustbox}

\title{Chapter 3 Section 4}
\author{Andrew Taylor}
\date{May 16 2022}
\newtheorem{theorem}{Theorem}
\newtheorem{problem}{Problem}
\newtheorem*{solution}{Solution}
\newcommand{\rref}[1]{\mathrm{rref \, #1}}
\newcommand{\im}[1]{\mathrm{im \, #1}}
\newcommand{\Span}[1]{\mathrm{span(#1)}}
\newcommand{\Rank}[1]{\mathrm{rank(#1)}}
\newcommand{\Coord}[2]{\left[ \vec{#1} \right]_{\mathfrak{#2}}}
\begin{document}
\maketitle

\begin{problem}
Let $\vec{v_{1}} = \begin{bmatrix} 1 \\ 1 \\ 1 \end{bmatrix}$ and $\vec{v_{2}} = \begin{bmatrix} 1 \\ 2 \\ 3 \end{bmatrix}$. Let $V = \Span{v_{1}, v_{2}}$. Is the vector $\vec{w} = \begin{bmatrix} 5 \\ 7 \\ 9 \end{bmatrix}$ on the plane V?
\end{problem}

\begin{solution}
If the vector $\vec{w}$ is on the plane V, then there exist some $x_{1}, x_{2} \in \mathbb{R}$ such that $\vec{w} = x_{1} \vec{v_{1}} + x_{2} \vec{v_{2}}$. This gives us the equations

\begin{align*}
x_{1} + x_{2} &= 5 \\
x_{1} + 2x_{2} &= 7 \\ 
x_{1} + 3x_{2} &= 9
\end{align*}

We can solve these equations using a matrix.

\begin{align*}
\left[ \begin{array}{cc|c}
1 & 1 & 5 \\
1 & 2 & 7 \\
1 & 3 & 9
\end{array} \right] \\
\left[ \begin{array}{cc|c}
1 & 1 & 5 \\
0 & 1 & 2 \\
0 & 2 & 4
\end{array} \right] \\
\left[ \begin{array}{cc|c}
1 & 1 & 5 \\
0 & 1 & 2 \\
0 & 0 & 0
\end{array} \right] \\
\left[ \begin{array}{cc|c}
1 & 0 & 3 \\
0 & 1 & 2 \\
0 & 0 & 0
\end{array} \right]
\end{align*}

This gives us $x_{1} = 3$ and $x_{2} = 2$. \\

Thus $\vec{w}$ is on the plane V because $\vec{w} = 3 \vec{v_{1}} + 2 \vec{v_{2}}$.

\end{solution}

\begin{problem}
Consider the basis $\mathfrak{B}$ of $\mathbb{R}^2$ consisting of the vectors $\vec{v_{1}} = \begin{bmatrix}3 \\ 1 \end{bmatrix}$ and $\vec{v_{2}} = \begin{bmatrix}-1 \\ 3 \end{bmatrix}$. 

\begin{itemize}
\item If $\vec{x} = \begin{bmatrix} 10 \\ 10 \end{bmatrix}$ find $\Coord{x}{B}$
\item If $\Coord{y}{B} = \begin{bmatrix} 2 \\ -1 \end{bmatrix}$ find $\vec{y}$
\end{itemize}
\end{problem}

\begin{solution}
We can find the coordinates of $\vec{x}$ with respect to $\mathfrak{B}$ by means of an equation.

\begin{align*}
\vec{x} &= c_{1} \vec{v_{1}} + c_{2} \vec{v_{2}} \\
\begin{bmatrix} 10 \\ 10 \end{bmatrix} &= c_{1} \begin{bmatrix}3 \\ 1 \end{bmatrix} + c_{2} \begin{bmatrix} -1 \\ 3 \end{bmatrix} \\
\begin{bmatrix} 10 \\ 10 \end{bmatrix} &= \begin{bmatrix}3 & -1 \\ 1 & 3 \end{bmatrix} \begin{bmatrix} c_{1} \\ c_{2} \end{bmatrix} \\
\end{align*}

We can solve this equation using elementary row operations.

\begin{align*}
\left( \begin{array}{cc|c}
3 & -1 & 10 \\
1 & 3 & 10
\end{array} \right) \\
\left( \begin{array}{cc|c}
0 & -10 & -20 \\
1 & 3 & 10
\end{array} \right) \\
\left( \begin{array}{cc|c}
0 & 1 & 2 \\
1 & 3 & 10
\end{array} \right) \\
\left( \begin{array}{cc|c}
0 & 1 & 2 \\
1 & 0 & 4
\end{array} \right) \\
\left( \begin{array}{cc|c}
1 & 0 & 4 \\
0 & 1 & 2 
\end{array} \right) \\
\end{align*}

By reducing the matrix, we find that $c_{1}  = 4$ and $c_{2} = 2$. Thus $\Coord{x}{B} = \begin{bmatrix} 4 \\ 2 \end{bmatrix}$. \\

We can also use an equation to solve for $\vec{y}$.

\begin{align*}
\vec{y} &= 2 \vec{v_{1}} - \vec{v_{2}} \\
&= 2 \begin{bmatrix}3 \\ 1 \end{bmatrix} - \begin{bmatrix} -1 \\ 3 \end{bmatrix} \\
&= \begin{bmatrix} 6 \\ 2 \end{bmatrix} - \begin{bmatrix} -1 \\ 3 \end{bmatrix} \\
&= \begin{bmatrix} 7 \\ -1 \end{bmatrix}
\end{align*}
\end{solution}

\end{document}