\documentclass{article}
\usepackage{amsmath, amssymb, amsthm, graphicx}
\usepackage[export]{adjustbox}

\title{Chapter 3 Section 4}
\author{Andrew Taylor}
\date{May 16 2022}
\newtheorem{theorem}{Theorem}
\newtheorem{problem}{Problem}
\newtheorem*{solution}{Solution}
\newcommand{\rref}[1]{\mathrm{rref \, #1}}
\newcommand{\im}[1]{\mathrm{im \, #1}}
\newcommand{\Span}[1]{\mathrm{span(#1)}}
\newcommand{\Rank}[1]{\mathrm{rank(#1)}}
\newcommand{\Coord}[2]{\Bigg[ \, \vec{#1} \, \Bigg]_{\mathfrak{#2}}}
\begin{document}
\maketitle

\begin{problem}
Let $\vec{v_{1}} = \begin{bmatrix} 1 \\ 1 \\ 1 \end{bmatrix}$ and $\vec{v_{2}} = \begin{bmatrix} 1 \\ 2 \\ 3 \end{bmatrix}$. Let $V = \Span{v_{1}, v_{2}}$. Is the vector $\vec{w} = \begin{bmatrix} 5 \\ 7 \\ 9 \end{bmatrix}$ on the plane V?
\end{problem}

\begin{solution}
If the vector $\vec{w}$ is on the plane V, then there exist some $x_{1}, x_{2} \in \mathbb{R}$ such that $\vec{w} = x_{1} \vec{v_{1}} + x_{2} \vec{v_{2}}$. This gives us the equations

\begin{align*}
x_{1} + x_{2} &= 5 \\
x_{1} + 2x_{2} &= 7 \\ 
x_{1} + 3x_{2} &= 9
\end{align*}

We can solve these equations using a matrix.

\begin{align*}
\left[ \begin{array}{cc|c}
1 & 1 & 5 \\
1 & 2 & 7 \\
1 & 3 & 9
\end{array} \right] \\
\left[ \begin{array}{cc|c}
1 & 1 & 5 \\
0 & 1 & 2 \\
0 & 2 & 4
\end{array} \right] \\
\left[ \begin{array}{cc|c}
1 & 1 & 5 \\
0 & 1 & 2 \\
0 & 0 & 0
\end{array} \right] \\
\left[ \begin{array}{cc|c}
1 & 0 & 3 \\
0 & 1 & 2 \\
0 & 0 & 0
\end{array} \right]
\end{align*}

This gives us $x_{1} = 3$ and $x_{2} = 2$. \\

Thus $\vec{w}$ is on the plane V because $\vec{w} = 3 \vec{v_{1}} + 2 \vec{v_{2}}$.

\end{solution}

\begin{problem}
Consider the basis $\mathfrak{B}$ of $\mathbb{R}^2$ consisting of the vectors $\vec{v_{1}} = \begin{bmatrix}3 \\ 1 \end{bmatrix}$ and $\vec{v_{2}} = \begin{bmatrix}-1 \\ 3 \end{bmatrix}$. 

\begin{itemize}
\item If $\vec{x} = \begin{bmatrix} 10 \\ 10 \end{bmatrix}$ find $\Coord{x}{B}$
\item If $\Coord{y}{B} = \begin{bmatrix} 2 \\ -1 \end{bmatrix}$ find $\vec{y}$
\end{itemize}
\end{problem}

\begin{solution}
We can find the coordinates of $\vec{x}$ with respect to $\mathfrak{B}$ by means of an equation.

\begin{align*}
\vec{x} &= c_{1} \vec{v_{1}} + c_{2} \vec{v_{2}} \\
\begin{bmatrix} 10 \\ 10 \end{bmatrix} &= c_{1} \begin{bmatrix}3 \\ 1 \end{bmatrix} + c_{2} \begin{bmatrix} -1 \\ 3 \end{bmatrix} \\
\begin{bmatrix} 10 \\ 10 \end{bmatrix} &= \begin{bmatrix}3 & -1 \\ 1 & 3 \end{bmatrix} \begin{bmatrix} c_{1} \\ c_{2} \end{bmatrix} \\
\end{align*}

We can solve this equation using elementary row operations.

\begin{align*}
\left( \begin{array}{cc|c}
3 & -1 & 10 \\
1 & 3 & 10
\end{array} \right) \\
\left( \begin{array}{cc|c}
0 & -10 & -20 \\
1 & 3 & 10
\end{array} \right) \\
\left( \begin{array}{cc|c}
0 & 1 & 2 \\
1 & 3 & 10
\end{array} \right) \\
\left( \begin{array}{cc|c}
0 & 1 & 2 \\
1 & 0 & 4
\end{array} \right) \\
\left( \begin{array}{cc|c}
1 & 0 & 4 \\
0 & 1 & 2 
\end{array} \right) \\
\end{align*}

By reducing the matrix, we find that $c_{1}  = 4$ and $c_{2} = 2$. Thus $\Coord{x}{B} = \begin{bmatrix} 4 \\ 2 \end{bmatrix}$. \\

We can also use an equation to solve for $\vec{y}$.

\begin{align*}
\vec{y} &= 2 \vec{v_{1}} - \vec{v_{2}} \\
&= 2 \begin{bmatrix}3 \\ 1 \end{bmatrix} - \begin{bmatrix} -1 \\ 3 \end{bmatrix} \\
&= \begin{bmatrix} 6 \\ 2 \end{bmatrix} - \begin{bmatrix} -1 \\ 3 \end{bmatrix} \\
&= \begin{bmatrix} 7 \\ -1 \end{bmatrix}
\end{align*}
\end{solution}

\begin{problem}
Let $\vec{v_{1}}$ and $\vec{v_{2}}$ be perpendicular unit vectors in $\mathbb{R}^3$. Let $\vec{v_{3}}$ be the cross product of $\vec{v_{1}}$ and $\vec{v_{2}}$, that is, $\vec{v_{3}} = \vec{v_{1}} \times \vec{v_{2}}$. We know from the properties of the cross product that $\vec{v_{3}}$ is perpendicular to $\vec{v_{1}}$ and $\vec{v_{2}}$. Thus the three vectors are linearly independent. The three vectors form a basis for $\mathbb{R}^3$.

\begin{enumerate}
\item What is $\vec{v_{1}} \times \vec{v_{3}}$?
\item Find the $\mathfrak{B}$-matrix of the linear transformation $T(x) = \vec{v_{1}} \times \vec{x}$.
\end{enumerate}
\end{problem}

\begin{solution}
$\vec{v_{1}} \times \vec{v_{3}} = -\vec{v_{2}}$. \\

The $\mathfrak{B}$-matrix is the matrix B such that \\

\begin{align*}
\Coord{T(x)}{B} = B \Coord{x}{B}
\end{align*}

%We know that $\Coord{x}{B} = \begin{bmatrix} \vec{x} \cdot \vec{v_{1}} \\ \vec{x} \cdot \vec{v_{2}} \\ \vec{x} \cdot \vec{v_{3}}\end{bmatrix}$.

We can find the coordinates of $\vec{x}$ with respect to $\mathfrak{B}$ by means of an equation.

\begin{align*}
\vec{x} &= c_{1} \vec{v_{1}} + c_{2} \vec{v_{2}} + c_{3} \vec{v_{3}} \\
&= \begin{bmatrix} \vec{v_{1}} & \vec{v_{2}} & \vec{v_{3}} \end{bmatrix} \begin{bmatrix} c_{1} \\ c_{2} \\ c_{3} \end{bmatrix}
\end{align*}
\end{solution}

Likewise, we have

\begin{align*}
T(x) &= \vec{v_{1}} \times \vec{x} \\
&= \vec{v_{1}} \times (c_{1} \vec{v_{1}} + c_{2} \vec{v_{2}} + c_{3} \vec{v_{3}}) \\
&= c_{1} (\vec{v_{1}} \times \vec{v_{1}}) + c_{2} (\vec{v_{1}} \times \vec{v_{2}}) + c_{3} (\vec{v_{1}} \times \vec{v_{3}}) \\
&= c_{2} v_{3} - c_{3} \vec{v_{2}} \\
&= \begin{bmatrix} \vec{v_{1}} & \vec{v_{2}} & \vec{v_{3}} \end{bmatrix} \begin{bmatrix} 0 \\ -c_{3} \\ c_{2} \end{bmatrix}
\end{align*}

Thus 

\begin{align*}
\Coord{x}{B} = \begin{bmatrix} c_{1} \\ c_{2} \\ c_{3} \end{bmatrix}
\end{align*} 

and 

\begin{align*}
\Coord{T(x)}{B} = \begin{bmatrix} 0 \\ -c_{3} \\ c_{2} \end{bmatrix}
\end{align*} 

Now let's find the matrix B such that 

\begin{align*}
\begin{bmatrix} 0 \\ -c_{3} \\ c_{2} \end{bmatrix} = B \begin{bmatrix} c_{1} \\ c_{2} \\ c_{3} \end{bmatrix}
\end{align*}

By inspection, we see that

\begin{align*}
B = \begin{bmatrix} 0 & 0 & 0 \\ 0 & 0 & -1 \\ 0 & 1 & 0 \end{bmatrix}
\end{align*}

Thus 

\begin{align*}
\Coord{T(x)}{B} = \begin{bmatrix} 0 & 0 & 0 \\ 0 & 0 & -1 \\ 0 & 1 & 0 \end{bmatrix} \Coord{x}{B}
\end{align*}

and the $\mathfrak{B}$-matrix of T is

\begin{align*}
B = \begin{bmatrix} 0 & 0 & 0 \\ 0 & 0 & -1 \\ 0 & 1 & 0 \end{bmatrix}
\end{align*}

\end{document}