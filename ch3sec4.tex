\documentclass{article}
\usepackage{amsmath, amssymb, amsthm, graphicx}
\usepackage[export]{adjustbox}

\title{Chapter 3 Section 4}
\author{Andrew Taylor}
\date{May 16 2022}
\newtheorem{theorem}{Theorem}
\newtheorem{problem}{Problem}
\newtheorem*{solution}{Solution}
\newcommand{\rref}[1]{\mathrm{rref \, #1}}
\newcommand{\im}[1]{\mathrm{im \, #1}}
\newcommand{\Span}[1]{\mathrm{span(#1)}}
\newcommand{\Rank}[1]{\mathrm{rank(#1)}}
\begin{document}
\maketitle

\begin{problem}
Let $\vec{v_{1}} = \begin{bmatrix} 1 \\ 1 \\ 1 \end{bmatrix}$ and $\vec{v_{2}} = \begin{bmatrix} 1 \\ 2 \\ 3 \end{bmatrix}$. Let $V = \Span{v_{1}, v_{2}}$. Is the vector $\vec{w} = \begin{bmatrix} 5 \\ 7 \\ 9 \end{bmatrix}$ on the plane V?
\end{problem}

\begin{solution}
If the vector $\vec{w}$ is on the plane V, then there exist some $x_{1}, x_{2} \in \mathbb{R}$ such that $\vec{w} = x_{1} \vec{v_{1}} + x_{2} \vec{v_{2}}$. This gives us the equations

\begin{align*}
x_{1} + x_{2} &= 5 \\
x_{1} + 2x_{2} &= 7 \\ 
x_{1} + 3x_{2} &= 9
\end{align*}

We can solve these equations using a matrix.

\begin{align*}
\left[ \begin{array}{cc|c}
1 & 1 & 5 \\
1 & 2 & 7 \\
1 & 3 & 9
\end{array} \right] \\
\left[ \begin{array}{cc|c}
1 & 1 & 5 \\
0 & 1 & 2 \\
0 & 2 & 4
\end{array} \right] \\
\left[ \begin{array}{cc|c}
1 & 1 & 5 \\
0 & 1 & 2 \\
0 & 0 & 0
\end{array} \right] \\
\left[ \begin{array}{cc|c}
1 & 0 & 3 \\
0 & 1 & 2 \\
0 & 0 & 0
\end{array} \right]
\end{align*}

This gives us $x_{1} = 3$ and $x_{2} = 2$. \\

Thus $\vec{w}$ is on the plane V because $\vec{w} = 3 \vec{v_{1}} + 2 \vec{v_{2}}$.

\end{solution}

\end{document}